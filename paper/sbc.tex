% Created 2023-11-09 Thu 16:37
% Intended LaTeX compiler: pdflatex
\documentclass[12pt]{article}
\usepackage[utf8]{inputenc}
\usepackage[T1]{fontenc}
\usepackage{graphicx}
\usepackage{longtable}
\usepackage{wrapfig}
\usepackage{rotating}
\usepackage[normalem]{ulem}
\usepackage{amsmath}
\usepackage{amssymb}
\usepackage{capt-of}
\usepackage{hyperref}
\usepackage[utf8]{inputenc}
\usepackage{sbc-template}
\usepackage{graphicx,url}
\address{Universidade do Sul de Santa Catarina (UNISUL)\\ Tubarão - SC - Brasil\\ anajuliabit@gmail.com}
\sloppy
\usepackage{biblatex}

\addbibresource{references.bib}
\author{Ana Julia Bittencourt Fogaça}
\date{\today}
\title{Uma análise classificatória em bugs encontrados em contratos inteligentes escritos em Solidity entre janeiro e setembro de 2023}
\hypersetup{
 pdfauthor={Ana Julia Bittencourt Fogaça},
 pdftitle={Uma análise classificatória em bugs encontrados em contratos inteligentes escritos em Solidity entre janeiro e setembro de 2023},
 pdfkeywords={},
 pdfsubject={Informações sobre as competições analisadas. HRF denota para High Risk Findings (bugs com severidade alta), nSLOC denota para número de linhas de código.},
 pdfcreator={Emacs 29.1 (Org mode 9.7)}, 
 pdflang={Portuges}}
\begin{document}

\maketitle

\section{Abstract}
\label{abstract}
\section{Resumo}
\label{resumo}
\section{Introdução}
\label{sec:orgcd9e81e}
Introduzida pela primeira vez em 2008 através do whitepaper do
Bitcoin\autocite{nakamotoBitcoinPeertoPeerElectronic}, a tecnologia blockchain é
reconhecida como uma megatendência em computação com o potencial para
transformar diversas indústrias\autocite{TechnologyTippingPoints}. Suas
características únicas de segurança, transparência e rastreabilidade têm
incentivado uma ampla variedade de setores a adotá-la para remodelar suas
operações essenciais. Até 2023, o valor de mercado das criptomoedas, o caso de
uso mais popular da blockchain até o momento, ultrapassou um trilhão de
dólares\autocite{CryptocurrencyStatistics20232023}. A aplicabilidade da
blockchain vai além das criptomoedas, abrangendo áreas como finanças,
gerenciamento de identidade, saúde e governança
eleitoral\autocite{BlockchainAdoptionsMaritime}.

A publicação do whitepaper do Ethereum em 2014 marcou um avanço significativo na
evolução da blockchain. Diferente do Bitcoin, que foi projetado inicialmente
como uma versão p2p de dinheiro
eletrônico\autocite{nakamotoBitcoinPeertoPeerElectronic}, o Ethereum introduziu a
noção revolucionária de contratos inteligentes\autocite{EthereumWhitepaper}. Essa
funcionalidade expandiu o alcance da tecnologia blockchain para novos setores. A
inovação do Ethereum reside em sua capacidade de suportar uma máquina virtual
que pode executar códigos em linguagens de programação \emph{Turing-complete}. No
entanto, como qualquer software, contratos inteligentes são desenvolvidos por
humanos e, portanto, estão sujeitos a erros. Em um ambiente de código aberto -
característica que é inerente à blockchains como o Ethereum, essas
vulnerabilidades se tornam alvos lucrativos para hackers. Somente no primeiro
trimestre de 2023, o Ethereum perdeu 320 milhões de dólares devido a ataques
cibernéticos\autocite{HereHowMuch}. Para mitigar esses riscos, projetos que rodam
em blockchains realizam auditorias antes de fazer o \emph{deploy} da aplicação. Existem
dos tipos de auditorias, as privadas, onde se é contratado uma empresa
especialista em auditoria de contratos inteligentes, e as públicas, realizadas
através de plataformas como a\autocite{Sherlock} e\autocite{Code4renaKeepingHigh} onde
qualquer individuo, empresa ou instituição pode participar e a recompensa é
ofertada individualmente para o primeiro participante que achar ou é dividida
entre todos os participantes que acharam o mesmo bug.


\#TODO mencionar pesquisas ja realizados
Com a crescente demanda por contratos inteligentes e uma expectativa de aumento
anual de 82,2\% de 2023 a 2030\autocite{SmartContractsMarket}, torna-se fundamental
compreender e categorizar as vulnerabilidades recentes. Neste artigo, analisamos
um conjunto de dados de 154 bugs extraídos de 31 competições realizadas entre
janeiro a setembro de 2023, através de duas plataformas de alta
reputação,\autocite{Sherlock} e\autocite{Code4renaKeepingHigh}. Nosso estudo busca
esclarecer questões críticas, como a dificuldade de detecção de diferentes tipos
de bugs, a prevalência de certas categorias de bugs em distintos protocolos, e a
relação entre as recentes vulnerabilidades exploradas por hackers e as
vulnerabilidades encontradas nas competições.
\section{Revisão bibliográfica}
\label{sec:orgcb9685f}

\subsection{Ethereum Blockchain}
\label{sec:org0911a1a}
Ethereum, conforme delineado no whitepaper por Vitalik Buterin et al., é uma
plataforma descentralizada que permite a construção de aplicações financeiras em
cima de uma infraestrutura de blockchain\autocite{EthereumWhitepaper}. A blockchain
é um sistema de registro distribuído e imutável que mantém um registro contínuo
de transações ou dados em blocos ligados por criptografia. Esse design assegura
a integridade e a veracidade dos dados, resistindo a alterações retroativas.
\subsection{Contratos inteligentes}
\label{sec:org5a1abae}
Contratos inteligentes são programas que rodam na blockchain do Ethereum,
permitindo que as partes cumpram acordos sem a necessidade de um intermediário.
Uma vez implantados, os contratos inteligentes não podem ser alterados, o que
exige que o código seja verificado para potenciais vulnerabilidades antes do
lançamento. Eles são fundamentais para a finança descentralizada e têm bilhões
de dólares em valor atrelados a eles\autocite{JCPFreeFullText}.
\subsection{Ethereum Virtual Machine}
\label{sec:org68c8ae8}
A Ethereum Virtual Machine (EVM) é o ambiente de execução para contratos
inteligentes na Ethereum. Funciona como uma camada global que pode executar
código de contrato inteligente em um contexto
descentralizado\autocite{woodETHEREUMSECUREDECENTRALISED}. Isso possibilita que os
desenvolvedores criem aplicações que funcionam exatamente conforme programadas,
sem qualquer possibilidade de fraude ou interferência de terceiros. A EVM é
isolada, significando que o código executado dentro dela não tem acesso ao
sistema de arquivos da rede, a outros contratos inteligentes, ou a qualquer
recurso externo. Esse isolamento garante um alto nível de segurança no
ecossistema Ethereum.
\subsection{Solidity}
\label{sec:orgb0eafbc}
Solidity é uma linguagem de programação de alto nível para a implementação de
contratos inteligentes e é fortemente tipada, suporta herança, bibliotecas e
tipos de usuário complexos\autocite{SoliditySolidity22}. Projetada para se alinhar
com a EVM, Solidity facilita o desenvolvimento de contratos inteligentes através
de uma sintaxe semelhante a JavaScript, tornando-a acessível a um amplo espectro
de programadores. Solidity, apesar de ser uma linguagem de alto nível com
características robustas, não está isenta de vulnerabilidades. Muitas delas
decorrem de uma desconexão entre a semântica da linguagem e a intuição dos
programadores, principalmente porque Solidity implementa características de
linguagens conhecidas, como JavaScript, de maneiras peculiares. Além disso, a
linguagem carece de construções específicas para lidar com aspectos do domínio
de blockchain, como a imprevisibilidade na ordem ou no atraso das etapas de
computação registradas publicamente na blockchain
\autocite{SolidityVulnerabilities}. Isso ressalta a importância de uma compreensão
aprofundada de Solidity ao desenvolver contratos inteligentes, para mitigar o
risco de vulnerabilidades de segurança.
\subsection{Finanças descentralizadas}
\label{sec:org4766323}
\begin{enumerate}
\item {\bfseries\sffamily TODO} explain DeFi
\label{sec:orgbbc37ba}
\end{enumerate}
\subsection{ERCs}
\label{sec:org95a8e82}
\section{Metodologia e perguntas da pesquisa}
\label{sec:org5c47fde}
\subsection{Coleta de dados}
\label{sec:orgc757d8f}
Foram identificados 145 bugs de alta severidade em 31 competições realizadas
entre janeiro e setembro de 2023, nas plataformas
Code4rena\autocite{Code4renaKeepingHigh} e Sherlock
\autocite{timeMostInterestingWeb32023}. As competições geralmente têm duração
aproximada de 7 dias e têm como objetivo a identificação de bugs antes do deploy
oficial. O montante total das recompensas distribuídas nas 31 competições ultrapassa a soma
de dois milhões de dólares e contam, em média, com 150 participantes por competição. Após a
etapa de participação, juízes — auditores de contratos inteligentes experientes
e selecionados pela comunidade — determinam o nível de severidade dos bugs
identificados. Bugs classificados como de alta severidade estão associados a
riscos significativos, como o roubo ou a perda de ativos digitais
\autocite{JudgingCriteria}.

\begin{center}
\begin{tabular}{lllrrrrr}
Plataforma & Categoria & Competição & Prêmio & HRF & nSLOC & Participantes & Data\\[0pt]
\hline
Code4rena & DAO & \href{https://code4rena.com/reports/2023-08-arbitrum}{Arbitrum security council election system} & 90500 & 1 & 2184 & 39 & 2023-09\\[0pt]
Code4rena & DAO & \href{https://code4rena.com/reports/2023-06-llama}{Llama} & 60500 & 2 & 2096 & 50 & 2023-07\\[0pt]
Code4rena & Stablecoin & \href{https://code4rena.com/reports/2023-06-lybra}{Lybra finance} & 60500 & 8 & 1762 & 136 & 2023-08\\[0pt]
Code4rena & Dexes & \href{https://code4rena.com/reports/2023-05-maia}{Maia DAO ecosystem} & 300500 & 35 & 10997 & 85 & 2023-05\\[0pt]
Code4rena & Yield & \href{https://code4rena.com/reports/2023-07-pooltogether\#wardens}{PoolTogether} & 121650 & 9 & 3324 & 117 & 2023-07\\[0pt]
Code4rena & Yield & \href{https://code4rena.com/reports/2023-08-pooltogether}{PoolTogether v5: part deux} & 42000 & 2 & 1001 & 45 & 2023-08\\[0pt]
Sherlock & Lending & \href{https://audits.sherlock.xyz/contests/75}{Ajna update} & 85600 & 6 & 5659 & 155 & 2023-06\\[0pt]
Sherlock & Yield Agreggator & \href{https://audits.sherlock.xyz/contests/41}{Blueberry} & 72500 & 10 &  & 284 & 2023-02\\[0pt]
Sherlock & Yield Agreggator & \href{https://audits.sherlock.xyz/contests/104/report}{Blueberry Update \#3} & 23600 & 5 & 3633 & 183 & 2023-08\\[0pt]
Sherlock & Opções & \href{https://audits.sherlock.xyz/contests/99}{Bond options} & 23600 & 2 & 885 & 153 & 2023-07\\[0pt]
Sherlock & Empréstimos & \href{https://audits.sherlock.xyz/contests/107}{Cooler update} & 17000 & 4 & 512 & 170 & 2023-08\\[0pt]
Sherlock & Dexes & \href{https://audits.sherlock.xyz/contests/97}{GFX labs} & 20400 & 2 & 716 & 106 & 2023-07\\[0pt]
Sherlock & Derivativos & \href{https://audits.sherlock.xyz/contests/74}{GMX} & 200000 & 5 & 10571 & 220 & 2023-04\\[0pt]
Sherlock & Lending & \href{https://audits.sherlock.xyz/contests/84}{Iron bank} & 67400 & 1 & 2241 & 271 & 2023-05\\[0pt]
Sherlock & Derivativos & \href{https://audits.sherlock.xyz/contests/79}{Perennial} & 122000 & 1 & 4063 & 220 & 2023-05\\[0pt]
Sherlock & Derivativos & \href{https://audits.sherlock.xyz/contests/106}{Perennial v2} & 125200 & 6 & 2494 & 252 & 2023-07\\[0pt]
Sherlock & Derivativos & \href{https://audits.sherlock.xyz/contests/85}{Symmetrical} & 91000 & 8 & 3553 & 233 & 2023-06\\[0pt]
Sherlock & Derivativos & \href{https://audits.sherlock.xyz/contests/108}{Symmetrical Update} & 27600 & 2 & 3921 & 52 & 2023-08\\[0pt]
Sherlock & Launchpad & \href{https://audits.sherlock.xyz/contests/100}{Tokensoft} & 21400 & 1 & 769 & 221 & 2023-07\\[0pt]
Sherlock & Stablecoin & \href{https://audits.sherlock.xyz/contests/73}{Unitas protocol} & 36400 & 1 & 1433 & 208 & 2023-06\\[0pt]
Code4rena & RWA & \href{https://code4rena.com/contests/2023-01-ondo-finance-contest}{Ondo finance} & 60500 & 1 & 4365 & 74 & 2023-01\\[0pt]
Sherlock & Índices & \href{https://audits.sherlock.xyz/contests/81}{Index coop} & 130600 & 2 & 4383 & 283 & 2023-05\\[0pt]
Sherlock & Stablecoin & \href{https://audits.sherlock.xyz/contests/82}{USSD} & 18200 & 3 & 402 & 224 & 2023-05\\[0pt]
Sherlock & RWA & \href{https://audits.sherlock.xyz/contests/98}{Dinari} & 16000 & 1 & 575 & 176 & 2023-07\\[0pt]
Sherlock & Dexes & \href{https://audits.sherlock.xyz/contests/88}{RealWagmi} & 33200 & 5 & 1080 & 203 & 2023-06\\[0pt]
Code4rena & DAO & \href{https://code4rena.com/reports/2023-07-nounsdao}{Nouns DAO} & 100000 & 1 & 9098 & 36 & 2023-07\\[0pt]
Sherlock & Dexes & \href{https://audits.sherlock.xyz/contests/89}{DODO v3} & 57800 & 5 & 2079 & 151 & 2023-06\\[0pt]
Sherlock & Derivativos & \href{https://audits.sherlock.xyz/contests/72}{Hubble Exchange} & 60000 & 3 & 1945 & 148 & 2023-06\\[0pt]
Code4rena & Stablecoin & \href{https://code4rena.com/contests/2023-06-angle-protocol-invitational}{Angle Protocol} & 52500 & 3 & 2276 & 5 & 2023-07\\[0pt]
Code4rena & Gestores de liquidez & \href{https://audits.sherlock.xyz/contests/86}{Arrakis} & 81400 & 2 & 2801 & 247 & 2023-06\\[0pt]
Sherlock & Dexes & \href{https://audits.sherlock.xyz/contests/95}{Unstoppable} & 36400 & 8 & 2035 & 130 & 2023-06\\[0pt]
\hline
TOTALS &  &  & 2255950 & 145 & 3095.1 & 157.32258 & \\[0pt]
\end{tabular}
\end{center}
\subsection{Categoria dos protocolos}
\label{sec:org1eef361}
Os protocolos analisados são exclusivamente do setor de Finanças Descentralizadas (DeFi) e englobam as subcategorias a seguir, seguindo a taxonomia de DefiLlama:

\begin{itemize}
\item Derivativos: Protocolos que oferecem instrumentos para negociações com alavancagem, permitindo aos usuários especular sobre os preços futuros de ativos e potencializar seus ganhos ou perdas.
\item \emph{Yield Farming}: Protocolos que incentivam os usuários a participar do staking ou a prover liquidez, recompensando-os por essas ações.
\item Agregadores de \emph{Yield}: Protocolos que maximizam os retornos ao combinar diferentes estratégias de \emph{yield farming}.
\item Opções: Protocolos que proporcionam o direito, mas não a obrigação, de comprar um ativo a um preço predeterminado em uma data futura.
\item DAOs (Organizações Autônomas Descentralizadas): Estruturas organizacionais emergentes que operam sem uma entidade centralizada, com decisões tomadas coletivamente pelos participantes.
\item Launchpads: Protocolos destinados a introduzir novos projetos e criptomoedas no mercado.
\item Índices: Protocolos que acompanham ou replicam o desempenho de um conjunto de ativos correlacionados.
\item DEXs (Trocas Descentralizadas): Protocolos que facilitam a troca de criptomoedas de maneira descentralizada.
\item RWAs (Ativos do Mundo Real): Protocolos que envolvem a tokenização de ativos tangíveis, como propriedades imobiliárias.
\item Stablecoins: Moedas digitais cujo valor é vinculado a moedas fiduciárias ou outros ativos, mantendo estabilidade por meio de mecanismos descentralizados.
\item Gestores de Liquidez: Protocolos que administram posições de liquidez em market makers automatizados com liquidez focalizada.
\item Empréstimos: Protocolos que facilitam o empréstimo e o empréstimo de uma variedade de ativos.
\end{itemize}
\subsection{Classificação dos Bugs}
\label{sec:org952703f}
A taxonomia utilizada foi uma junção da taxonomia apresentada por \autocite{zhangDemystifyingExploitableBugs2023a} e as tags apresentadas por Solodit\autocite{Solodit_contentReport_tagsMd}.

\begin{itemize}
\item O: Fora do Escopo
\begin{itemize}
\item Não é possível acessar o código-fonte do projeto.
\item Bugs que ocorrem em componentes fora da cadeia (off-chain).
\item Contratos inteligentes escritos em outra linguagem.
\end{itemize}
\item C01: Manipulação do Mempool / Vulnerabilidades de Front-Running
\begin{itemize}
\item Ataques sandwich \#TODO
\item Flash loan exploits \#TODO
\end{itemize}
\item C02: Ataque de Reentrada
Vulnerabilidades de reentrância ocorrem quando chamadas a contratos externos são feitas antes de atualizações de estado internas, permitindo que um adversário chame recursivamente o contrato, explorando o estado inconsistente.
\item C03: Atualizações de Estado Errôneas.
Ausência de atualização de estado ou atualizações incorretas, por exemplo, uma atualização de estado que não deveria estar presente.
\item C04: \emph{Hardcoded} configuração
Prática de incorporar valores ou parâmetros fixos diretamente no código-fonte de um contrato inteligente. Isso pode representar um risco à segurança se a configuração precisar ser dinâmica ou adaptável.
\item C05: Escalada de Privilégios e Problemas de Controle de Acesso.
\begin{itemize}
\item Funções privilegiadas que podem ser chamadas por qualquer um ou a qualquer momento.
\item Fundos de usuários que podem ficar presos devido a código de retirada ausente ou incorreto.
\end{itemize}
\item C06: Matemática Incorreta / Contabilidade Errônea.
Matemática Incorreta refere-se a um problema potencial onde operações matemáticas dentro de um contrato inteligente são implementadas incorretamente, levando a cálculos imprecisos.
Entre eles incluem:
\begin{itemize}
\item Ordem de cálculo incorreta.
\item Retorno de um valor inesperado que desvia da semântica esperada especificada para o contrato.
\item Cálculos realizados com números incorretos (por exemplo, x = a + b ==> x = a + c, precisões incorretas).
\item Outros erros de contabilidade (por exemplo, x = a + b ==> x = a - b).
\item Underflow/overflow.
\end{itemize}
\item C07: Lógica de Negócios Quebrada.
Envolvem falhas na lógica de negócios ou protocolos de um contrato inteligente, onde a implementação corresponde à intenção do desenvolvedor, mas a lógica subjacente é inerentemente falha.
\begin{itemize}
\item Sequências de invocação de função inesperadas ou ausentes (por exemplo, chamadas externas a contratos dependentes, sequências exploráveis que levam à realocação ou manipulação maliciosa de fundos).
\item Condições inesperadas do ambiente ou contrato
\item Argumentos de função inesperados.
\end{itemize}
\item C08: Bugs Específicos da Implementação do Contrato.
Estes bugs são difíceis de categorizar em outras categorias.
\item C09: Falta de Proteção Contra Replay de Assinatura
\begin{itemize}
\item Nonce ausente \#TODO
\item Colisão de hash.
\end{itemize}
\item C10: Verificação Ausente.
Falha crítica no código de um contrato inteligente onde uma condição ou validação necessária não é implementada adequadamente.
\item C11: DoS (Negação de Serviço).
Vulnerabilidades de DoS ocorrem quando um atacante pode explorar um contrato de maneira que o torne irresponsivo ou significativamente menos eficiente. Esta categoria inclui casos que não são bem descritos por outra classe e onde a consequência primária é o encerramento do contrato ou ineficiência operacional.
\item C12: Validação de Dados
Vulnerabilidades de validação de dados surgem quando um contrato inteligente não verifica ou saneia adequadamente as entradas, especialmente aquelas provenientes de fontes não confiáveis. Esta falta de validação pode levar a consequências não intencionais e potencialmente prejudiciais nas operações do contrato.
\item C13: Correspondência de Lista Branca/Lista Negra.
O contrato inteligente lida inadequadamente com endereços baseados em listas predefinidas.
\item C14: Arrays.
Vulnerabilidades relacionadas a arrays podem surgir quando os desenvolvedores não manuseiam adequadamente os índices de arrays ou falham em validar as entradas dos usuários.
\begin{itemize}
\item leituras/escritas fora dos limites
\item problemas na exclusão
\item problemas com o redimensionamento de arrays.
\end{itemize}
\end{itemize}
\subsection{Perguntas da pesquisa}
\label{sec:org5bf671c}
\begin{itemize}
\item Q1: Que tipo de vulnerabilidade é mais difícil de ser encontrada por auditores?
\item Q2: Que categoria de protocolo apresenta mais presença de bugs?
\item Q3: Os auditores frequentemente perdem tipos específicos de bugs que são posteriormente explorados?
\item Q4: Qual é o impacto financeiro médio de diferentes tipos de vulnerabilidades?
\item Q5: Como a complexidade do contrato inteligente afeta a probabilidade de encontrar bugs?
\item Q6: Qual a relação entre categoria de bugs e os diferente tipos de protocolos?
\end{itemize}
\subsection{Dados coletados}
\label{sec:orgb915b33}
\subsection{Resultados}
\label{sec:org527b341}
\section{Referências}
\label{sec:org19da17f}
\printbibliography
\end{document}