% Created 2023-11-02 Thu 18:51
% Intended LaTeX compiler: pdflatex
\documentclass[12pt]{article}
\usepackage[utf8]{inputenc}
\usepackage[T1]{fontenc}
\usepackage{graphicx}
\usepackage{longtable}
\usepackage{wrapfig}
\usepackage{rotating}
\usepackage[normalem]{ulem}
\usepackage{amsmath}
\usepackage{amssymb}
\usepackage{capt-of}
\usepackage{hyperref}
\usepackage[utf8]{inputenc}
\usepackage{sbc-template}
\usepackage{graphicx,url}
\address{Universidade do Sul de Santa Catarina (UNISUL)\\ Tubarão - SC - Brasil\\ anajuliabit@gmail.com}
\sloppy
\usepackage{biblatex}

\addbibresource{references.bib}
\author{Ana Julia Bittencourt Fogaça}
\date{\today}
\title{Uma análise classificatória em bugs encontrados em contratos inteligentes escritos em Solidity entre janeiro e setembro de 2023}
\hypersetup{
 pdfauthor={Ana Julia Bittencourt Fogaça},
 pdftitle={Uma análise classificatória em bugs encontrados em contratos inteligentes escritos em Solidity entre janeiro e setembro de 2023},
 pdfkeywords={},
 pdfsubject={Informações sobre as competições analisadas. HRF denota para High Risk Findings (bugs com severidade alta), nSLOC denota para número de linhas de código.},
 pdfcreator={Emacs 29.1 (Org mode 9.7)}, 
 pdflang={Portuges}}
\begin{document}

\maketitle

\section{Abstract}
\label{abstract}
\section{Resumo}
\label{resumo}
\section{Introdução}
\label{sec:orge06e943}
Introduzida pela primeira vez em 2008 através do whitepaper do
Bitcoin\autocite{nakamotoBitcoinPeertoPeerElectronic}, a tecnologia blockchain é
reconhecida como uma megatendência em computação com o potencial para
transformar diversas indústrias\autocite{TechnologyTippingPoints}. Suas
características únicas de segurança, transparência e rastreabilidade têm
incentivado uma ampla variedade de setores a adotá-la para remodelar suas
operações essenciais. Até 2023, o valor de mercado das criptomoedas, o caso de
uso mais popular da blockchain até o momento, ultrapassou um trilhão de
dólares\autocite{CryptocurrencyStatistics20232023}. A aplicabilidade da
blockchain vai além das criptomoedas, abrangendo áreas como finanças,
gerenciamento de identidade, saúde e governança
eleitoral\autocite{BlockchainAdoptionsMaritime}.

A publicação do whitepaper do Ethereum em 2014 marcou um avanço significativo na
evolução da blockchain. Diferente do Bitcoin, que foi projetado inicialmente
como uma versão p2p de dinheiro
eletrônico\autocite{nakamotoBitcoinPeertoPeerElectronic}, o Ethereum introduziu a
noção revolucionária de contratos inteligentes\autocite{EthereumWhitepaper}. Essa
funcionalidade expandiu o alcance da tecnologia blockchain para novos setores. A
inovação do Ethereum reside em sua capacidade de suportar uma máquina virtual
que pode executar códigos em linguagens de programação \emph{Turing-complete}. No
entanto, como qualquer software, contratos inteligentes são desenvolvidos por
humanos e, portanto, estão sujeitos a erros. Em um ambiente de código aberto que
é inerente à blockchain, essas vulnerabilidades se tornam alvos lucrativos para
hackers. Somente no primeiro trimestre de 2023, o Ethereum perdeu 320 milhões de
dólares devido a ataques cibernéticos\autocite{HereHowMuch}. Para mitigar esses
riscos, plataformas blockchain frequentemente oferecem recompensas financeiras
em troca de vulnerabilidades encontradas através de competições realizadas em
plataformas.

Com a crescente demanda por contratos inteligentes e uma expectativa de aumento
anual de 82,2\% de 2023 a 2030\autocite{SmartContractsMarket}, torna-se fundamental
compreender e categorizar as vulnerabilidades recentes. Neste artigo, analisamos
um conjunto de dados de 154 bugs extraídos de 31 competições realizadas entre
janeiro a setembro de 2023, através de duas plataformas de alta
reputação,\autocite{Sherlock} e\autocite{Code4renaKeepingHigh}. Nosso estudo busca
esclarecer questões críticas, como a dificuldade de detecção de diferentes tipos
de bugs, a prevalência de certas categorias de bugs em distintos protocolos, e a
relação entre as recentes vulnerabilidades exploradas por hackers e as
vulnerabilidades encontradas nas competições.
\section{Revisão bibliográfica}
\label{sec:org6f62092}

\subsection{Ethereum Blockchain}
\label{sec:orge064e98}
Ethereum, conforme delineado no whitepaper por Vitalik Buterin et al., é uma
plataforma descentralizada que permite a construção de aplicações financeiras em
cima de uma infraestrutura de blockchain\autocite{EthereumWhitepaper}. A blockchain
é um sistema de registro distribuído e imutável que mantém um registro contínuo
de transações ou dados em blocos ligados por criptografia. Esse design assegura
a integridade e a veracidade dos dados, resistindo a alterações retroativas.
\subsection{Contratos inteligentes}
\label{sec:org9a9c683}
Contratos inteligentes são programas que rodam na blockchain do Ethereum,
permitindo que as partes cumpram acordos sem a necessidade de um intermediário.
Uma vez implantados, os contratos inteligentes não podem ser alterados, o que
exige que o código seja verificado para potenciais vulnerabilidades antes do
lançamento. Eles são fundamentais para a finança descentralizada e têm bilhões
de dólares em valor atrelados a eles\autocite{JCPFreeFullText}.
\subsection{Ethereum Virtual Machine}
\label{sec:org79d41b8}
A Ethereum Virtual Machine (EVM) é o ambiente de execução para contratos
inteligentes na Ethereum. Funciona como uma camada global que pode executar
código de contrato inteligente em um contexto
descentralizado\autocite{woodETHEREUMSECUREDECENTRALISED}. Isso possibilita que os
desenvolvedores criem aplicações que funcionam exatamente conforme programadas,
sem qualquer possibilidade de fraude ou interferência de terceiros. A EVM é
isolada, significando que o código executado dentro dela não tem acesso ao
sistema de arquivos da rede, a outros contratos inteligentes, ou a qualquer
recurso externo. Esse isolamento garante um alto nível de segurança no
ecossistema Ethereum.
\subsection{Solidity}
\label{sec:org188de1d}
Solidity é uma linguagem de programação de alto nível para a implementação de
contratos inteligentes e é fortemente tipada, suporta herança, bibliotecas e
tipos de usuário complexos\autocite{SoliditySolidity22}. Projetada para se alinhar
com a EVM, Solidity facilita o desenvolvimento de contratos inteligentes através
de uma sintaxe semelhante a JavaScript, tornando-a acessível a um amplo espectro
de programadores. Solidity, apesar de ser uma linguagem de alto nível com
características robustas, não está isenta de vulnerabilidades. Muitas delas
decorrem de uma desconexão entre a semântica da linguagem e a intuição dos
programadores, principalmente porque Solidity implementa características de
linguagens conhecidas, como JavaScript, de maneiras peculiares. Além disso, a
linguagem carece de construções específicas para lidar com aspectos do domínio
de blockchain, como a imprevisibilidade na ordem ou no atraso das etapas de
computação registradas publicamente na blockchain
\autocite{SolidityVulnerabilities}. Isso ressalta a importância de uma compreensão
aprofundada de Solidity ao desenvolver contratos inteligentes, para mitigar o
risco de vulnerabilidades de segurança.
\subsection{Protocolos}
\label{sec:orgc39d5f8}
\begin{enumerate}
\item {\bfseries\sffamily TODO} explain DEFI
\label{sec:org99d69db}
\end{enumerate}
\subsection{ERCs}
\label{sec:org05bd3e7}
\section{Metodologia e perguntas da pesquisa}
\label{sec:orgca30adb}
\subsection{Coleta de dados}
\label{sec:org90b18b3}
Foram coletados 145 bugs encontrados em 31 competições que aconteceram entre Janeiro a Setembro de 2023 nas plataformas Code4rena\autocite{Code4renaKeepingHigh} e \autocite{Sherlock}. Todos foram classificados com severidade alta -  ativos podem ser roubados, perdidos ou compromissados\autocite{JudgingCriteria}. Geralmente as competições duram em torno de 7 dias e o objetivo é capturar bugs antes de realizar o deploy oficial. As somas das premiações das 31 competições analisadas ultrapassam dois milhões de dólares, e em média aproximadamente participam 150 participantes por competição.
\begin{center}
\begin{tabular}{lllrrrrr}
Plataforma & Categoria & Competição & Prêmio & HRF & nSLOC & Participantes & Data\\[0pt]
\hline
Code4rena & DAO & \href{https://code4rena.com/reports/2023-08-arbitrum}{Arbitrum security council election system} & 90500 & 1 & 2184 & 39 & 2023-09\\[0pt]
Code4rena & DAO & \href{https://code4rena.com/reports/2023-06-llama}{Llama} & 60500 & 2 & 2096 & 50 & 2023-07\\[0pt]
Code4rena & Stablecoin & \href{https://code4rena.com/reports/2023-06-lybra}{Lybra finance} & 60500 & 8 & 1762 & 136 & 2023-08\\[0pt]
Code4rena & Dexes & \href{https://code4rena.com/reports/2023-05-maia}{Maia DAO ecosystem} & 300500 & 35 & 10997 & 85 & 2023-05\\[0pt]
Code4rena & Yield & \href{https://code4rena.com/reports/2023-07-pooltogether\#wardens}{PoolTogether} & 121650 & 9 & 3324 & 117 & 2023-07\\[0pt]
Code4rena & Yield & \href{https://code4rena.com/reports/2023-08-pooltogether}{PoolTogether v5: part deux} & 42000 & 2 & 1001 & 45 & 2023-08\\[0pt]
Sherlock & Lending & \href{https://audits.sherlock.xyz/contests/75}{Ajna update} & 85600 & 6 & 5659 & 155 & 2023-06\\[0pt]
Sherlock & Yield Agreggator & \href{https://audits.sherlock.xyz/contests/41}{Blueberry} & 72500 & 10 &  & 284 & 2023-02\\[0pt]
Sherlock & Yield Agreggator & \href{https://audits.sherlock.xyz/contests/104/report}{Blueberry Update \#3} & 23600 & 5 & 3633 & 183 & 2023-08\\[0pt]
Sherlock & Options & \href{https://audits.sherlock.xyz/contests/99}{Bond options} & 23600 & 2 & 885 & 153 & 2023-07\\[0pt]
Sherlock & Lending & \href{https://audits.sherlock.xyz/contests/107}{Cooler update} & 17000 & 4 & 512 & 170 & 2023-08\\[0pt]
Sherlock & Dexes & \href{https://audits.sherlock.xyz/contests/97}{GFX labs} & 20400 & 2 & 716 & 106 & 2023-07\\[0pt]
Sherlock & Derivatives & \href{https://audits.sherlock.xyz/contests/74}{GMX} & 200000 & 5 & 10571 & 220 & 2023-04\\[0pt]
Sherlock & Lending & \href{https://audits.sherlock.xyz/contests/84}{Iron bank} & 67400 & 1 & 2241 & 271 & 2023-05\\[0pt]
Sherlock & Derivatives & \href{https://audits.sherlock.xyz/contests/79}{Perennial} & 122000 & 1 & 4063 & 220 & 2023-05\\[0pt]
Sherlock & Derivatives & \href{https://audits.sherlock.xyz/contests/106}{Perennial v2} & 125200 & 6 & 2494 & 252 & 2023-07\\[0pt]
Sherlock & Derivatives & \href{https://audits.sherlock.xyz/contests/85}{Symmetrical} & 91000 & 8 & 3553 & 233 & 2023-06\\[0pt]
Sherlock & Derivatives & \href{https://audits.sherlock.xyz/contests/108}{Symmetrical Update} & 27600 & 2 & 3921 & 52 & 2023-08\\[0pt]
Sherlock & Launchpad & \href{https://audits.sherlock.xyz/contests/100}{Tokensoft} & 21400 & 1 & 769 & 221 & 2023-07\\[0pt]
Sherlock & Stablecoin & \href{https://audits.sherlock.xyz/contests/73}{Unitas protocol} & 36400 & 1 & 1433 & 208 & 2023-06\\[0pt]
Code4rena & RWA & \href{https://code4rena.com/contests/2023-01-ondo-finance-contest}{Ondo finance} & 60500 & 1 & 4365 & 74 & 2023-01\\[0pt]
Sherlock & Indexes & \href{https://audits.sherlock.xyz/contests/81}{Index coop} & 130600 & 2 & 4383 & 283 & 2023-05\\[0pt]
Sherlock & Stablecoin & \href{https://audits.sherlock.xyz/contests/82}{USSD} & 18200 & 3 & 402 & 224 & 2023-05\\[0pt]
Sherlock & RWA & \href{https://audits.sherlock.xyz/contests/98}{Dinari} & 16000 & 1 & 575 & 176 & 2023-07\\[0pt]
Sherlock & Dexes & \href{https://audits.sherlock.xyz/contests/88}{RealWagmi} & 33200 & 5 & 1080 & 203 & 2023-06\\[0pt]
Code4rena & DAO & \href{https://code4rena.com/reports/2023-07-nounsdao}{Nouns DAO} & 100000 & 1 & 9098 & 36 & 2023-07\\[0pt]
Sherlock & Dexes & \href{https://audits.sherlock.xyz/contests/89}{DODO v3} & 57800 & 5 & 2079 & 151 & 2023-06\\[0pt]
Sherlock & Derivatives & \href{https://audits.sherlock.xyz/contests/72}{Hubble Exchange} & 60000 & 3 & 1945 & 148 & 2023-06\\[0pt]
Code4rena & Stablecoin & \href{https://code4rena.com/contests/2023-06-angle-protocol-invitational}{Angle Protocol} & 52500 & 3 & 2276 & 5 & 2023-07\\[0pt]
Code4rena & Liquidity manager & \href{https://audits.sherlock.xyz/contests/86}{Arrakis} & 81400 & 2 & 2801 & 247 & 2023-06\\[0pt]
Sherlock & Dexes & \href{https://audits.sherlock.xyz/contests/95}{Unstoppable} & 36400 & 8 & 2035 & 130 & 2023-06\\[0pt]
\hline
TOTALS &  &  & 2255950 & 145 & 3095.1 & 157.32258 & \\[0pt]
\end{tabular}
\end{center}
\subsection{Categoria dos protocolos}
\label{sec:org85933bb}
As competições referem a diferente protocolos do setor de Finanças Descentralizadas que podem ser classificados em:
\begin{itemize}
\item Derivativos: Protocolos para apostas com alavancagem, permitindo que os usuários especulem sobre preços futuros de ativos com a possibilidade de ampliar seus ganhos (ou perdas).
\item \emph{Yield}: Protocolos que recompensam os usuários por fazer \emph{"stake"} ou fornecer liquidez em suas plataformas.
\item Agregadores de Yield: Protocolos que combinam e otimizam o rendimento de diferentes fontes de yield.
\item Opções: Protocolos que oferecem o direito de comprar um ativo por um preço fixo em uma data futura.
\item DAO (Organização Autônoma Descentralizada): Estruturas legais emergentes sem um corpo governante central, onde as decisões são tomadas de forma coletiva pelos membros.
\item Launchpad: Protocolos que lançam novos projetos e criptomoedas no mercado.
\item Índices: Protocolos que rastreiam ou criam o desempenho de um grupo de ativos relacionados.
\item Dexes (Trocas Descentralizadas): Protocolos que permitem a troca ou negociação de criptomoedas.
\item RWA (Ativos do Mundo Real): Protocolos envolvendo a tokenização de ativos do mundo real, como imóveis.
\item Stablecoin: Moedas estáveis atreladas ao dólar ou outras moedas fiduciárias através de mecanismos descentralizados.
\item Gestores de Liquidez: Protocolos que gerenciam posições de liquidez em formadores de mercado automatizados com liquidez concentrada.
\item Empréstimos: Protocolos que possibilitam aos usuários emprestar e tomar emprestado ativos diversos.
\end{itemize}
\subsection{Perguntas da pesquisa}
\label{sec:org3522bc8}
\begin{itemize}
\item Q1: Que tipo de vulnerabilidade é mais difícil de ser encontrada por auditores?
\item Q2: Que categoria de protocolo apresenta mais presença de bugs?
\item Q3: Os auditores frequentemente perdem tipos específicos de bugs que são posteriormente explorados?
\item Q4: Qual é o impacto financeiro médio de diferentes tipos de vulnerabilidades?
\item Q5: Como a complexidade do contrato inteligente afeta a probabilidade de encontrar bugs?
\item Q6: Qual a relação entre categoria de bugs e os diferente tipos de protocolos?
\end{itemize}
\subsection{Classificação dos bugs}
\label{sec:orge955fcc}
\begin{itemize}
\item O: Out-of-scope
\begin{itemize}
\item We cannot access the source code of the project.
\item Bugs that occur in off-chain components
\item Smart contracts are written in another language
\end{itemize}
\item C01: Mempool Manipulation / Front-Running Vulnerabilities, (e.g sandwich attacks, flash-loan exploits)
\item C02: Reentry attack
Reentrancy vulnerabilities happen when external contract calls are made before internal state updates, allowing an adversary to recursively call back into the contract, exploiting the inconsistent state.
\item C03: Erroneous state updates.
Missing state update, or incorrect state updates, e.g., a state update that should not be there.
\item C04: Hardcoded Setting - refers to the practice of embedding fixed values or parameters directly into the source code of a smart contract. This can pose a security risk if the setting needs to be dynamic or adaptable.
\item C05: Privilege escalation and access control issues.
\begin{itemize}
\item Privileged functions can be called by anyone or at any time.
\item User funds can get locked due to missing/wrong withdraw code
\end{itemize}
\item C06: Wrong Math / Erroneous accounting.
Wrong Math refers to a potential issue where mathematical operations within a smart contract are implemented incorrectly, leading to inaccurate calculations.
\begin{itemize}
\item Incorrect calculating order.
\item Returning an unexpected value that deviates from the expected semantics specified for the contract.
\item Calculations performed with incorrect numbers (e.g., x = a + b ==> x = a + c, incorrect precisions).
\item Other accounting errors (e.g., x = a + b ==> x = a - b).
\item Underflow/overflow
\end{itemize}
\item C07: Broken business logic
\end{itemize}
Logic vulnerabilities involve flaws in the business logic or protocols of a smart contract, where the implementation matches the developer's intention, but the underlying logic is inherently flawed.
\begin{itemize}
\item Unexpected or missing function invocation sequences (e.g., external calls to dependent contracts,  exploitable sequences leading to malicious fund reallocation or manipulation).
\item Unexpected environment or contract conditions (e.g., ChainLink returning outdated data or significant slippage occurring).
\item Unexpected function arguments.
\end{itemize}
\begin{itemize}
\item C08: Contract implementation-specific bugs. These bugs are difficult to categorize into others categories.
\item C09: Lack of signature replay protection, e.g missing nonce, hash collision
\item C10: Missing check.
Missing Check refers to a critical oversight in a smart contract's code where a necessary condition or validation is not properly implemented.
\item C11: lack of segregation between users funds
\item C12: Data validation: Data validation vulnerabilities arise when a smart contract does not adequately verify or sanitize inputs, especially those from untrusted sources. This lack of validation can lead to unintended and potentially harmful consequences within the contract’s operations.
\item C13: Whitelist/Blacklist Match
Whitelist/Blacklist Match refers to a potential vulnerability where a smart contract improperly handles addresses based on predefined lists.
\item C14: Arrays
Vulnerabilities related to arrays can arise when developers do not properly handle array indices or fail to validate user inputs.
\end{itemize}
would typically be reserved for vulnerabilities that directly arise from mishandling or misinterpreting arrays in the code. For example, if there were out-of-bound reads/writes, deletion mishaps, or issues with array resizing
\begin{itemize}
\item C15: DoS
Denial of Service (DoS) vulnerabilities occur when an attacker can exploit a contract in a way that makes it unresponsive or significantly less efficient. This category includes cases that are not well described by another class and where the primary consequence is contract shut-down or operational inefficiency.
\end{itemize}
\subsection{Dados coletados}
\label{sec:orge007ca2}
\subsection{Desenvolvimento}
\label{sec:orgf04edbc}
\subsection{Categorias}
\label{sec:org1c41fb4}
\subsection{Dificuldade}
\label{sec:org153d176}
\section{Referências}
\label{sec:org516d785}
\printbibliography
\end{document}